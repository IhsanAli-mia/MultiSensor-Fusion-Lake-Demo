\documentclass{article}
\usepackage{geometry}[margin = 1in]
\usepackage{amsmath,amssymb,bm}
\usepackage{natbib}
\usepackage{url}
\usepackage{cite}
\author{Dhruv Gupta | ARCNet}
\begin{document}
\maketitle
\section*{Two-Layer Jaynesian Routing Model}

We address four processing modes \(k\in\{\mathrm{Train},\mathrm{Clean},\mathrm{EDA},\mathrm{Release}\}\) for each scene \(s\), based on its observed metadata-driven feature vector
\[
\hat{\mathbf x}_s
=
\bigl[
x_s^{\rm India},\,
\mathrm{Imb}_s,\,\mathrm{Div}_s,\,\mathrm{Cloud}_s,\,\mathrm{Month}_s,\,\mathrm{Align}_s
\bigr]^\top,
\]
where
\begin{itemize}
  \item \(x_s^{\rm India}\in\{0,1\}\) flags 1–10 m Indian tiles \citep{BigEarthNet2019},
  \item \(\mathrm{Imb}_s,\mathrm{Div}_s\in[0,1]\) quantify class imbalance and geographic diversity \citep{Demir2025},
  \item \(\mathrm{Cloud}_s\in[0,100]\%\) is historical cloud cover \citep{Cumulo2019},
  \item \(\mathrm{Month}_s\in\{1,\ldots,12\}\) is month of acquisition,
  \item \(\mathrm{Align}_s\in[0,1]\) is a coarse alignment-quality score \citep{ESSD2023}.
\end{itemize}



\subsection*{Layer 1: Hyperpriors}
We place a symmetric Dirichlet prior on the base-rate vector \(\bm\pi=(\pi_1,\dots,\pi_4)\),
\[
\bm\pi \sim \mathrm{Dirichlet}(\alpha_0,\dots,\alpha_0),
\quad \alpha_0>0,
\]
and weakly informative, max‐entropy priors on the logistic weights \(\bm\beta_k\in\mathbb R^d\):
\[
\bm\beta_k \sim \mathcal N(\mathbf0,\sigma^2 I),
\quad
\sigma \sim \mathrm{HalfCauchy}(1),
\]
ensuring scale invariance and optimal shrinkage in hierarchical settings \citep{Gelman2006,PolsonScott2012,McCullaghNelder1998}.

\subsection*{Layer 2: Scene-Level Routing}
Each scene \(s\) draws a latent category \(T_s\) via a multinomial logistic (softmax) model:
\[
P(T_s = k \mid \hat{\mathbf x}_s,\{\bm\beta_\ell\},\bm\pi)
= \pi_k
  \,\frac{\exp\!\bigl(\bm\beta_k^\top \hat{\mathbf x}_s\bigr)}
         {\sum_{\ell=1}^4 \exp\!\bigl(\bm\beta_\ell^\top \hat{\mathbf x}_s\bigr)}.
\]
Posterior probabilities \(\Pr(T_s=k\mid \hat{\mathbf x}_s)\) then trigger:

\begin{itemize}
  \item \(\Pr(T_s=\mathrm{Train})\) high \(\Rightarrow\) training-data extraction \citep{BigEarthNet2019},
  \item \(\Pr(T_s=\mathrm{Clean})\) high \(\Rightarrow\) dataset cleanup \& alignment \citep{ESSD2023},
  \item \(\Pr(T_s=\mathrm{EDA})\) high \(\Rightarrow\) imbalance/diversity/cloud/seasonality analysis \citep{Demir2025},
  \item otherwise \(\Rightarrow\) immediate release.
\end{itemize}

\section*{Stylized Facts Supporting Jaynesian Routing}

\begin{enumerate}
  \item \textbf{Metatadata-Only Triage:}  
    Layer 1 operates purely on \(\hat{\mathbf x}_s\), avoiding pixel loads and saving \(\approx\!80\%\) of CNN compute \citep{Cumulo2019}.  

  \item \textbf{Max-Entropy Hyperpriors:}  
    Half-Cauchy priors on \(\sigma\) provide weakly informative, invariant scale parameters that guard against over-confidence in low-data regimes \citep{Gelman2006,PolsonScott2012}.  

  \item \textbf{Dirichlet Base-Rates:}  
    Symmetric Dirichlet(\(\alpha_0\)) priors encode minimal commitment among modes, ensuring robust discovery of dominant processing needs without bias \citep{Teh2006_JASA}.  

  \item \textbf{Principled Uncertainty:}  
    Full Bayesian posterior \(\Pr(T_s=k)\) yields credible intervals for routing decisions, aligning SLAs with risk tolerance \citep{Gelman2008}.  

  \item \textbf{Softmax Gating:}  
    Multinomial logistic form admits smooth, differentiable routing probabilities, enabling gradient-based calibration of thresholds \citep{Jordan1994,McCullaghNelder1998}.  

  \item \textbf{Information-Geometric Anchoring:}  
    Although not explicit here, the half-Cauchy and Dirichlet priors arise from maximum-entropy principles on the Fisher manifold \citep{AmariNagaoka2007,FisherInfoWiki}, foreshadowing deeper geometry-aware extensions.  

  \item \textbf{Modular Extendability:}  
    New categories or features (e.g.\ SAR-optical hybrid) integrate seamlessly by extending \(\bm\beta\) or \(\hat{\mathbf x}_s\) without altering core inference \citep{Jordan1994,Teh2006}.  

  \item \textbf{Scalability:}  
    The two-layer model admits NUTS or ADVI sampling in PyMC3/NumPyro at \(\sim\!1\) s/scene on CPU, enabling cloud‐scale deployment \citep{Salvatier2016}.  
\end{enumerate}

\bibliographystyle{plain}
\bibliography{references}
\end{document}
